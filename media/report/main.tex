\documentclass{scrartcl}

\usepackage[utf8]{inputenc}
\usepackage{amsmath}
\usepackage{amsthm}
\usepackage{amssymb}
\usepackage{geometry}
\usepackage{mathtools}
\usepackage{cancel}
\usepackage{xfrac}
\usepackage{siunitx}                    % for scientific notation \num{}
\usepackage{authblk}                    % for authors
\usepackage{tikz}                       % for circled {}
\usepackage{systeme}                    % for system of equations bracket
\usepackage{verbatim}                   % for comment
\usepackage{xfp}                        % for floating point operations in macros
\usepackage{graphicx}                   % for cropping the images
\usepackage{dsfont}                     % for doublestroke fonts
\usepackage{float}
\usepackage{hyperref}                   % hyperlink
\usepackage[english]{babel}
\usepackage[square, numbers]{natbib}
\usepackage[nottoc]{tocbibind}          % Includes "References" in the table of contents

\bibliographystyle{abbrvnat}

\hypersetup{
	colorlinks=true, 
	citecolor=orange,
}

\AtBeginDocument{\renewcommand{\bibname}{References}}

\title{Video Prediction}
\subtitle{Independent Work Report (MAE 340, Spring 2021)}
\author{Matthew Coleman}
\date{April 26, 2022}

\begin{document}

\maketitle

\vspace{8cm}
\Large
\textit{This project represents my own work, in accordance with the University regulations.} \\
\hspace*{\fill} \large /s/Matthew Coleman
\normalsize

\newpage
\tableofcontents
\newpage

\section{Introduction}
\label{sec:intro}

While humans cannot perfectly predict the future, they are indeed capable of
predicting near events to some extent, and this knowledge greatly aids them in
planning out their actions, such as which movements to take to reach a goal. To
some extent, this ability to forecast the future is a direct result of an
understanding of causality that is learned through observation and interaction
\cite{human_learning_sequences}.

A great amount of human predictions are, of course, entirely erroneous and
result in some consequences, but even the humans least adept at predicting
likely outcomes are still masters of predicting the future in some respects.
For example, humans have a good sense for where a car will move in the street,
or which direction a pedestrian may continue walking. Even a young child can
predict where to toss a football to a moving receiver, and even this small
knowledge reveals an infinite wisdom compared to the most advanced video
prediction methods.

The task of video prediction is comprised of several open challenges in
computer vision; it uses some of the most recent model architectures that have
been developed and it even deals directly with an impossible task altogether,
which is to predict the future. Although it is a particularly confusing task,
however, it also has the potential for immense impact and immediate practical
applications, such as in autonomous driving \cite{eg_self_driving}, video
interpolation \cite{eg_video_interp} and most interesting in the context of
this report, robotic control systems \cite{eg_robot_control}.

% On one hand, some situations are very predictable, such as the motion of a
% pendulum swinging in a clock, or the spinning of a merry-go-round. These are
% mostly deterministic events, that is, they have a high likelihood of continuing
% in a predictable way. Other situations are more unpredictable, such as the
% moves and strikes of an opponent boxer, or the paths of people walking on a
% busy street. These events are mostly stochastic, meaning that they have only a
% lowlikelihood of proceeding in any given way, and the longer the sequence goes
% on, the more unlikely it will be that a person--or a machine learning
% model--can predict what will happen next.

% The task of video prediction in computer vision is a self-supervising task that
% involves splitting a sequence of video frames into an input data and label
% pair. The task is self-supervising because the first half of the data is used
% as the input and the second half is used as the label, that is, the model
% directly computes a loss function between it's own output and the actual ground
% truth frames from the original sequence. In effect, the model learns solely
% from real-world data, without any human intervention in the data-labeling
% process.

\section{Prediction Model Families}
\label{sec:families}

In addition to convolution, modern video prediction models tend to adopt a few
canonical architectures, which are each very well-researched and developed.
Although they are typically implemented slightly differently in practice, the
overarching themes are still in effect, and many of these model paradigms are
combined or adapted for alternative tasks. The best video prediction models use
them in conjunction with each other to create a larger network of interleaving
models. 

Of these are recurrent models, which are trained on sequences of data,
generative models, which are trained to approximate the conditional probability
$p(x y)$ in order to generate images in the same distribution as ground-truth
images.

\subsection{Recurrent Models}
\label{subsec:recurrent}

\subsection{Generative Models}
\label{subsec:generative}

\section{Convolutional LSTM}
\label{sec:conv_lstm}

\section{FutureGAN}
\label{sec:futuregan}

\newpage
\section{}
\label{}

\newpage
\section{}
\label{}

\newpage
\section{}
\label{}

\newpage
\section{}
\label{}


% ------------------------------------------------------------------------------
% References 
% ------------------------------------------------------------------------------

\newpage
\bibliography{main}
\newpage

\end{document}

