\documentclass{scrartcl}

\usepackage[utf8]{inputenc}
\usepackage{amsmath}
\usepackage{amsthm}
\usepackage{amssymb}
\usepackage{geometry}
\usepackage{mathtools}
\usepackage{cancel}
\usepackage{xfrac}
\usepackage{siunitx}                    % for scientific notation \num{}
\usepackage{authblk}                    % for authors
\usepackage{tikz}                       % for circled {}
\usepackage{systeme}                    % for system of equations bracket
\usepackage{verbatim}                   % for comment
\usepackage{xfp}                        % for floating point operations in macros
\usepackage{graphicx}                   % for cropping the images
\usepackage{dsfont}                     % for doublestroke fonts
\usepackage{float}
\usepackage{hyperref}                   % hyperlink
\usepackage[english]{babel}
\usepackage[square, numbers]{natbib}
\usepackage[nottoc]{tocbibind}          % Includes "References" in the table of contents

\bibliographystyle{abbrvnat}

\hypersetup{
	colorlinks=true, 
	citecolor=orange,
}

\AtBeginDocument{\renewcommand{\bibname}{References}}

\title{Video Prediction}
\subtitle{Independent Work Report (MAE 340, Spring 2021)}
\author{Matthew Coleman}
\date{April 26, 2022}

\begin{document}

\maketitle

\vspace{8cm}
\Large
\textit{This project represents my own work, in accordance with the University regulations.} \\
\hspace*{\fill} \large /s/Matthew Coleman
\normalsize

\newpage
\tableofcontents
\newpage

\section{Introduction}
\label{sec:intro}

While humans cannot perfectly predict the future, they are indeed capable of
predicting near events to some extent, and this knowledge greatly aids them in
planning out their actions, such as which movements to take to reach a goal. To
some extent, this ability to forecast the future is a direct result of an
understanding of causality that is learned through observation and interaction
\cite{human_learning_sequences}.

A great amount of human predictions are, of course, entirely erroneous and
result in some consequences, but even the humans least adept at predicting
likely outcomes are still masters of predicting the future in some respects.
For example, humans have a good sense for where a car will move in the street,
or which direction a pedestrian may continue walking. Even a young child can
predict where to toss a football to a moving receiver, and even this small
knowledge reveals an infinite wisdom compared to the most advanced video
prediction methods.

The task of video prediction is comprised of several open challenges in
computer vision; it uses some of the most recent model architectures that have
been developed and it even deals directly with an impossible task altogether,
which is to predict the future. Although it is a particularly confusing task,
however, it also has the potential for immense impact and immediate practical
applications, such as in autonomous driving \cite{eg_self_driving}, video
interpolation \cite{eg_video_interp} and most interesting in the context of
this report, robotic control systems \cite{eg_robot_control}.

% On one hand, some situations are very predictable, such as the motion of a
% pendulum swinging in a clock, or the spinning of a merry-go-round. These are
% mostly deterministic events, that is, they have a high likelihood of continuing
% in a predictable way. Other situations are more unpredictable, such as the
% moves and strikes of an opponent boxer, or the paths of people walking on a
% busy street. These events are mostly stochastic, meaning that they have only a
% lowlikelihood of proceeding in any given way, and the longer the sequence goes
% on, the more unlikely it will be that a person--or a machine learning
% model--can predict what will happen next.

% The task of video prediction in computer vision is a self-supervising task that
% involves splitting a sequence of video frames into an input data and label
% pair. The task is self-supervising because the first half of the data is used
% as the input and the second half is used as the label, that is, the model
% directly computes a loss function between it's own output and the actual ground
% truth frames from the original sequence. In effect, the model learns solely
% from real-world data, without any human intervention in the data-labeling
% process.

\section{The Task of Video Prediction}
\label{sec:task}

The task of video prediction is to 

\section{Families of Prediction Models}
\label{sec:families}

Modern video prediction models tend to adopt canonical architectures.

\subsection{Recurrent Models}
\label{subsec:recurrent}

Recurrent neural networks (RNNs) consist of a network of nodes that stretches
over some sequential information, typically in the form of a time sequence
(Which is the case in video prediction). Each node will perform another
learning technique on the data in sequence (This could be a convolution
operation, or a linear layer, for example) and output its own activation.
Commonly, this is implemented not with a network of individual nodes but rather
in the form of a feedback loop over a single node, which passes some data
embedding forward through the network to itself in a loop (a hidden state, or
variable), only taking in new data from the original input sequence at each
step. In this way, an RNN equipped with convolution is capable of learning from
time-varying information while preserving spatio-temporal relations (That is,
relationships in the data that exist over space, such as the shape of a
person's leg and hip, as well as relationships in the data that exist over
time, such as the motion of a person walking, will be preserved in the
final activations of the network).

The outputs of each node are then used for other purposes, depending on the
task, and the result is then backpropagated against a loss function. In machine
learning parlance, this is referred to as backpropagation through time (BPTT),
since a gradient must be computed in the input sequence's reverse order, i.e.,
through time, and in the case of an RNN implemented with a feedback system,
this gradient must be computed with respect to the input data at each time step
and then added to the weights of the single node in sum. 

\subsection{Generative Models}
\label{subsec:generative}

\section{Convolutional LSTM}
\label{sec:conv_lstm}

\section{FutureGAN}
\label{sec:futuregan}

\newpage
\section{}
\label{}

\newpage
\section{}
\label{}

\newpage
\section{}
\label{}

\newpage
\section{}
\label{}


% ------------------------------------------------------------------------------
% References 
% ------------------------------------------------------------------------------

\newpage
\bibliography{main}
\newpage

\end{document}

